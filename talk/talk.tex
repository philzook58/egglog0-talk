\documentclass{beamer}
\usepackage{listings}
\usepackage{bussproofs}
%Information to be included in the title page:
\title{Logging an Egg:} 
\subtitle{Datalog on E-Graphs}   
\author{Philip Zucker}
% \email{pzucker@draper.com} 
\institute{Draper Laboratory}
\date{June 13, 2022}

\begin{document}

\frame{\titlepage}


\begin{frame}[fragile]
\frametitle{Datalog : Databases, Logic, and Proofs.}

  $$\forall x,y. edge(x,y) \implies path(x,y)$$
   \begin{prooftree}
        \AxiomC{edge(x,y)}
        \UnaryInfC{path(x,y)}
        \end{prooftree}
\begin{lstlisting}
        path(X, Y) :- edge(X, Y).
        path(X, Y) :- path(X, Z), edge(Z, Y).
        \end{lstlisting} 
        Search right hand side in database. Insert left side. Repeat.
\end{frame}

\begin{frame}
    \frametitle{Datalog: Applications}
    
    \begin{itemize}
        \item Graph Problems % Recursion is necessary. Breadth first search
        \item Worklist Algorithms
        \item You can program in it. Explicit control flow
        \item Mutually Recursive Analyses
        \item Program Analysis %\footnote{\url{https://souffle-lang.github.io/tutorial\#data-flow-analysis-example}}
            \begin{itemize}
                \item Doop \footnote{\url{https://bitbucket.org/yanniss/doop/src/master/}}
                \item DDisasm \footnote{\url{https://github.com/GrammaTech/ddisasm}}
            \end{itemize}
    \end{itemize}
\end{frame}


\begin{frame}
    \frametitle{Datalog vs Prolog}
    \begin{itemize}
        \item Pattern matching vs Unification
        \item Top down vs Bottom up
        \item Complete vs Incomplete Search
        \item Tabling
    \end{itemize}
\end{frame}

\begin{frame}
\frametitle{E-Graphs}
\begin{itemize}
   \item Datastructure for terms and equalities
   \item Egg \footnote{\url{https://egraphs-good.github.io/}}: Efficient Rust library
   \item Term Rewriting $ ?a + 0 \rightarrow ?a$
   \item An E-Graph Never Forgets
   \item Maximize sharing up and down.
  % \item Bipartite graph of enodes and eclasses
   \item Rule ordering
   \item $?a + (?b + ?c) = (?a + ?b) + ?c$. $?a + -?a = 0$
   \item $x + (-x + 10)$
\end{itemize}
\end{frame}

\begin{frame}
    \frametitle{Applications of E-Graphs}
    \begin{itemize}
       \item SMT and other Theorem Proving
        \item Compiler Optimization. PEG \footnote{\url{https://rosstate.org/publications/eqsat/}}
        \item Herbie \footnote{\url{https://herbie.uwplse.org/}}
        \item Query Optimization
        \item Szalinski - CAD
       \item YOGO \footnote{\url{https://www.jameskoppel.com/files/papers/yogo-preprint.pdf}}
    \end{itemize}
\end{frame}





\begin{frame}[fragile]
    \frametitle{Egglog0\footnote{\url{https://www.philipzucker.com/egglog/}}}
    \begin{itemize}
        \item E-Graphs are a Database
        \item The database holds terms and equality relation
        \item Supports ordinary datalog with terms
        \item Pattern variables bind to eclasses
        \item Rules: query using RHS (e-matching multipattern), instantiate and insert LHS
        \item Special equality $=$ is E-graph Equality
        \begin{lstlisting}
            add(Y,X) = E :- add(X,Y) = E.
        \end{lstlisting}
        \item Queries e-match and return all results.
        \begin{lstlisting}
            ?- add(succ(zero),succ(Y)) = Z.
        \end{lstlisting}
    \end{itemize}


\end{frame}

\begin{frame}
    \frametitle{Egg Multipatterns}
    \begin{itemize}
        \item Upstreamed to egg\footnote{\url{https://github.com/egraphs-good/egg/pull/168}}.
        \item Multipatterns vs Guards %: Multipatterns can bind new variables.
        \item Threads e-matching compiler env binding between patterns.
        %pattern names to runtime env slots. Multipatterns threads it between patterns
    \end{itemize}
\end{frame}


\begin{frame}
    \frametitle{Demo}
% Maybe I should 
\end{frame}

\begin{frame}[fragile]
    \frametitle{Example: Injectivity}
    \begin{itemize}
        \item $\forall a, b. f(a) = f(b) \implies a = b$
        %\item $\forall a, b. a != b \implies f(a) != f(b)$
        \item Example: Constructors, Negation, constant addition
        \item Unification
    \end{itemize}
    \begin{lstlisting}
        X = Y, Xs = Ys :- cons(X,Xs) = cons(Y,Ys).
        X = Y :- X + Z = Y + Z.
    \end{lstlisting}
\end{frame}

\begin{frame}[fragile]
    \frametitle{Example: Memory Simplification\footnote{\url{https://www.philipzucker.com/egglog/?example=mem.pl}}}
    \begin{itemize}
     %   \item Disequality reasoning and injectivity
        \item Alias Analysis + Simplification
        \item SMTlib theory of arrays %/ McCarthy
        \item Many SMT theories are expressible as Horn Clauses (side conditions)
    \end{itemize}
    \begin{lstlisting}[
        basicstyle=\tiny, %or \small or \footnotesize etc.
    ]
//select grabs stored value
V <- select(A ,store(A, V, Mem)).

//select ignores different addresses
select(A1,Mem) = E :- select(A1,store(A2,V,Mem)) = E, neq(A1,A2).

//non aliasing writes commute
store(A2,V2, store(A1,V1,Mem)) = E :- store(A1,V1, store(A2,V2,Mem)) = E, neq(A1,A2).

//Aliasing Writes destroy old value.
store(A, V1, Mem) <- store(A, V1, store(A,V2,Mem)).
\end{lstlisting}
\end{frame}

\begin{frame}[fragile]
    \frametitle{Example: Equation Solving}
    \begin{itemize}
        \item Isolation
        \item Extract terms without variables
    \end{itemize}
    \begin{lstlisting}
        sub(Z,X) = Y :- add(X,Y) = Z
    \end{lstlisting}
\end{frame}

\begin{frame}[fragile]
    \frametitle{Example: Reflection}
    \begin{itemize}
        \item Hypothetical reasoning
        \item Boolean algebraic reasoning
    \end{itemize}
    \begin{lstlisting}
        A = B :- true = eq(A,B).
        true = eq(A,B) :- A = B.    
    \end{lstlisting}
\end{frame}

\begin{frame}[fragile]
    \frametitle{Example: Uniqueness Quantification\footnote{\url{https://www.philipzucker.com/egglog/?example=cat1.pl}}}
    \begin{itemize}
        \item Common in universal constructions in category theory
        \item Skolemize existentials $\forall x, P(x) \implies \exists y, Q(x,y)$  becomes
                $\forall x, P(x) \implies Q(x,f(x))$
        \item Uniqueness Property $\forall a, b. P(a) \land P(b) \implies a = b$
    \end{itemize}
   % $ \forall x_1, \ldots, x_n, \phi(x_1, \ldots , x_n) \implies  \exists y_1, \ldots, y_m,  \psi(x_1,\ldots,x_n, y_1, \ldots, y_m)  $
        % ALT: $$ \forall \{x_n\}, \phi(\{x_n\}) \implies \exists \{y_m\},  \psi(\{x_n\}, \{y_m\}) $$
    %    can be converted into an axiom where $y_1 \ldots y_m$ are $n$-arity globally fresh function symbols.
        
     %    \begin{lstlisting}
     %   psi(X1,...,Xn,y1(X1,...,Xn),...,ym(X1,...,Xn)) :- phi(X1, ... ,Xn)
     %   \end{lstlisting}
\end{frame}

\begin{frame}
    \frametitle{Related Work}
    \begin{itemize} 
        \item Relational E-Matching \footnote{\url{https://arxiv.org/abs/2108.02290}}
        \item SMT Multipatterns
        \item Souffle Egg \footnote{\url{https://www.hytradboi.com/2022/writing-part-of-a-compiler-in-datalog}}
        \item Egg-lite
    \end{itemize}
\end{frame}

%\begin{frame}
%    \frametitle{Work to do}
%    \begin{itemize}
%        \item Semi-Naive
%        \item Efficient Indexing
%        \item Harrop
%    \end{itemize}
%\end{frame}

\begin{frame}
    \frametitle{Questions?}
    \begin{itemize}
        \item Thanks to Yihong Zhang, Remy Yisu
        Wang, Max Willsey, Zachary Tatlock, Alessandro Cheli, Cody
        Roux, James Fairbanks, and Evan Patterson for their helpful
        discussions.
        \item DARPA Grant 
       % \item Refs
    \end{itemize}
\end{frame}




\end{document}